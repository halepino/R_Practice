% Options for packages loaded elsewhere
\PassOptionsToPackage{unicode}{hyperref}
\PassOptionsToPackage{hyphens}{url}
%
\documentclass[
]{article}
\usepackage{amsmath,amssymb}
\usepackage{lmodern}
\usepackage{ifxetex,ifluatex}
\ifnum 0\ifxetex 1\fi\ifluatex 1\fi=0 % if pdftex
  \usepackage[T1]{fontenc}
  \usepackage[utf8]{inputenc}
  \usepackage{textcomp} % provide euro and other symbols
\else % if luatex or xetex
  \usepackage{unicode-math}
  \defaultfontfeatures{Scale=MatchLowercase}
  \defaultfontfeatures[\rmfamily]{Ligatures=TeX,Scale=1}
\fi
% Use upquote if available, for straight quotes in verbatim environments
\IfFileExists{upquote.sty}{\usepackage{upquote}}{}
\IfFileExists{microtype.sty}{% use microtype if available
  \usepackage[]{microtype}
  \UseMicrotypeSet[protrusion]{basicmath} % disable protrusion for tt fonts
}{}
\makeatletter
\@ifundefined{KOMAClassName}{% if non-KOMA class
  \IfFileExists{parskip.sty}{%
    \usepackage{parskip}
  }{% else
    \setlength{\parindent}{0pt}
    \setlength{\parskip}{6pt plus 2pt minus 1pt}}
}{% if KOMA class
  \KOMAoptions{parskip=half}}
\makeatother
\usepackage{xcolor}
\IfFileExists{xurl.sty}{\usepackage{xurl}}{} % add URL line breaks if available
\IfFileExists{bookmark.sty}{\usepackage{bookmark}}{\usepackage{hyperref}}
\hypersetup{
  pdftitle={Political Affiliation and Vaccine Hesitancy},
  pdfauthor={Holly Figueroa},
  hidelinks,
  pdfcreator={LaTeX via pandoc}}
\urlstyle{same} % disable monospaced font for URLs
\usepackage[margin=1in]{geometry}
\usepackage{graphicx}
\makeatletter
\def\maxwidth{\ifdim\Gin@nat@width>\linewidth\linewidth\else\Gin@nat@width\fi}
\def\maxheight{\ifdim\Gin@nat@height>\textheight\textheight\else\Gin@nat@height\fi}
\makeatother
% Scale images if necessary, so that they will not overflow the page
% margins by default, and it is still possible to overwrite the defaults
% using explicit options in \includegraphics[width, height, ...]{}
\setkeys{Gin}{width=\maxwidth,height=\maxheight,keepaspectratio}
% Set default figure placement to htbp
\makeatletter
\def\fps@figure{htbp}
\makeatother
\setlength{\emergencystretch}{3em} % prevent overfull lines
\providecommand{\tightlist}{%
  \setlength{\itemsep}{0pt}\setlength{\parskip}{0pt}}
\setcounter{secnumdepth}{-\maxdimen} % remove section numbering
\ifluatex
  \usepackage{selnolig}  % disable illegal ligatures
\fi

\title{Political Affiliation and Vaccine Hesitancy}
\author{Holly Figueroa}
\date{5/24/2021}

\begin{document}
\maketitle

\hypertarget{introduction}{%
\subsection{Introduction}\label{introduction}}

Since the onset of the Covid-19 pandemic, scientific and medical
communities around the globe have worked at record speed to provide the
public with information, guidance, and cures in the form of vaccines. As
the United States works to vaccinate citizens, the willingness of
citizens to do so is a barrier. The US has recorded nearly 33 million
cases of Covid-19 and nearly 600,000 deaths due the virus. Recent
research estimates the loss in the United States to be much higher,
around 900,000 deaths(Estimation of Total Mortality Due to COVID-19,
2021). While these facts would intuitively suggest people were eager to
vaccinate, data from US Census Bureau suggests many people are not ready
to vaccinate. Here, measures for people's willingness to vaccinate will
be explored with measures of personal impacts and hardships during the
pandemic. Understanding how experiences of income loss, housing
instability, and delayed medical care may shed light into the minds of
those more or less willing to vaccinate. Although vaccination is not a
new area distrust for many in the United States, government leadership
has added a powerful political element to the 2020 pandemic. Research
has since demonstrated the party affiliation of government leaders
correlates with citizens engagement in voluntary measures, such as
social distancing (Grossman, 2020). Therefore, election data will also
be explored for possible connections to vaccination views.

\hypertarget{the-problem--increasing-vaccination-participation}{%
\subsection{The Problem -Increasing Vaccination
Participation}\label{the-problem--increasing-vaccination-participation}}

Historically, vaccination efforts to protect the public have been
voluntary and have relied on public messaging and marketing to gain
participation(Schwartz, 2012). With marketing as the key method for
gaining cooperation in Covid-19 vaccinations, learning who to market-to
is an important step requiring analysis. Without care to understand who
is and who is not willing to vaccinate, any campaign to gain maximum
participation will fall short. This research seeks to find correlations
that may exist in the data available regarding community willingness to
vaccinate, racial demographics, hardship during the pandemic, and
political affiliation. Specifically, these questions include the
following:

\begin{itemize}
\tightlist
\item
  Does a relationship exist between race and vaccination hesitancy?
\item
  Does a relationship exist between and willingness to vaccinate and
  housing instability?
\item
  Does a relationship exist between willingness to vaccinate and income
  instability?
\item
  Does a relationship exist between delayed medical care and willingness
  to vaccinate?
\item
  Does a relationship exist between party support and willingness to
  vaccinate?
\end{itemize}

\hypertarget{approach---identifying-features-of-the-vaccine-hesitant-population}{%
\subsection{Approach - Identifying Features of the Vaccine-Hesitant
Population}\label{approach---identifying-features-of-the-vaccine-hesitant-population}}

To answer the questions above, two data sets were created in order to
perform correlational analyses. One data set combined responses from the
Census Bureau and 2020 Presidential Election outcomes at the state
level. Data from the Census Bureau is based on the direct responses of
the Household Pulse Survey (HPS) where individuals were asked about
their willingness to vaccinate, once a vaccine was available to them. In
this format, any correlations are made to ``willingness to vaccinate''.
The Census Bureau data also includes measures of impact from Covid-19.
Impacts include reports of anticipated income loss, anticipated eviction
or home foreclosure, and delayed medical treatment during the pandemic.
These variables combined will allow analyses to explore vaccination
views and any relationships to state party support and hardships during
the pandemic.

The second data set was created combining data from the CDC and 2020
Presidential Election outcomes at the county level. The CDC data
includes measures for ``vaccine hesitancy'' as opposed to
``willingness''. So, these terms will be used in relation to the data
sets involved, when applicable. In addition to hesitancy measures, the
CDC data also contains racial demographics at the county level. These
variables will allow a more nuanced level of analyses of hesitancy and
relationships to county party support and race.

\textbf{Measuring Party Influence}

As research has shown voluntary behaviors for precautions were muted
along republican party lines, focus was given to the republican party.
In both data sets, election results were used to gain a measure of
republican party support for the 2020 presidential race. Total votes
cast for president were calculated for each state and county,
respectively. From here, the percentage of total votes for republican
incumbent, Donald Trump, was obtained.

\hypertarget{analysis}{%
\subsection{Analysis}\label{analysis}}

\textbf{County Distribution of Vaccine Hesitancy}

To begin analyses, measures of ``hesitancy'' estimates calculated by the
CDC were checked for normality. The plot below illustrates the density
of CDC estimates across counties. The CDC data offer two measures for
hesitancy, ``hesitant'' and ``strongly hesitant''. At a glance it can be
seen that estimates of ``strongly hesitant'' are relatively normal in
distribution. It can also be seen that ``strongly hesitant'' estimates
are represented in a smaller range which occupies the lower end of
percentages. In short, county estimations of ``strongly hesitant'' were
generally lower percentages, peaking around eight percent. The density
of ``hesitant'' estimates across counties, was also near normal in
distribution. It can be seen the milder estimate of ``hesitant'' versus
``strongly hesitant'' occupies a wider range and more gradual incline to
it's peak. In short, percentages of ``hesitant'' estimates were higher
in relation. These characteristics also suggest measures of the
``hesitant'' county populations have greater variance and standard
deviation.

\includegraphics[width=0.7\linewidth]{Vaccine_Hesitancy_Final.FigueroaHollly_files/figure-latex/Density Plot for Hesitancy Responses -1}

When distributions for both ``hesitant'' and ``strongly hesitant'' are
tested for normality through use of q-q plots, the degree of normality
is more visible. Estimates of ``hesitant'' populations are rather
closely aligned to the line of normality. Estimates of ``strongly
hesitant'' populations are shown to curve upwards at the ends. This is
characteristic of the wider curve viewed in the initial density plot.
These deviations from the line, represent higher frequencies than normal
at either ends of the curve. As these deviations are relatively, mild,
these variables may be treated as normal. \newline

\includegraphics[width=0.5\linewidth]{Vaccine_Hesitancy_Final.FigueroaHollly_files/figure-latex/QQ-Plots For County Hesitancy-1}
\includegraphics[width=0.5\linewidth]{Vaccine_Hesitancy_Final.FigueroaHollly_files/figure-latex/QQ-Plots For County Hesitancy-2}

\textbf{County Distributions of Race}

Distributions of race across counties was explored. Despite having a
very large sample, the qq-plots below show that race is not normally
distributed across county populations. This is evidenced by the strong
curving away from the line of normality plotted along the observation
points, indicating a heavy positive skew for non white groups. A
negative skew is evidenced by the inverse curving from the line in the
q-plot for the white population. Non-normality was found for all race
variables. For this reason, analyses including variables of race will
require methods that do not assume normality.\\
\newline

\includegraphics[width=0.4\linewidth]{Vaccine_Hesitancy_Final.FigueroaHollly_files/figure-latex/CDC race variable normality tests-1}
\includegraphics[width=0.4\linewidth]{Vaccine_Hesitancy_Final.FigueroaHollly_files/figure-latex/CDC race variable normality tests-2}

\includegraphics[width=0.4\linewidth]{Vaccine_Hesitancy_Final.FigueroaHollly_files/figure-latex/contd..CDC race variable normality tests-1}
\includegraphics[width=0.4\linewidth]{Vaccine_Hesitancy_Final.FigueroaHollly_files/figure-latex/contd..CDC race variable normality tests-2}

\newline

\textbf{County Distribution of Election Results - Republican Party
Support}

The distribution of votes awarded to republican candidate, Donald Trump,
was not normal. The percentages across counties was heavily, positively
skewed, at 32.81 and had a positive measure of kurtosis at 1300.99. For
this reason, analyses conducted must not assume normality. \newline

\includegraphics[width=0.48\linewidth]{Vaccine_Hesitancy_Final.FigueroaHollly_files/figure-latex/Election Data County Level-1}
\includegraphics[width=0.48\linewidth]{Vaccine_Hesitancy_Final.FigueroaHollly_files/figure-latex/Election Data County Level-2}

\textbf{State Distribution of ``Willingness'' to Vaccinate}

State level data from the Census Bureau offers a single measure for
``willingness'' to vaccinate. When the distribution of all the state's
percentages is plotted for density, we see a normal curve. At a glance,
the curve indicates the average percent of citizens willing to vaccinate
is below 50\%. When the same data is tested for normality in a qq-plot,
we see again, the data is relatively normal in distribution. \newline

\includegraphics[width=0.5\linewidth]{Vaccine_Hesitancy_Final.FigueroaHollly_files/figure-latex/State Data Willing Density-1}
\includegraphics[width=0.5\linewidth]{Vaccine_Hesitancy_Final.FigueroaHollly_files/figure-latex/State Data Willing Density-2}

\textbf{State Distributions of Covid-19-Impacted Stability Variables}

Responses from the Census Bureau's Pulse Survey regarding hardships
during the pandemic were found to be relatively normal. The density
plots below illustrate imperfect curves, however, measures for skewness
and kurtosis were found to be within acceptable range. Expected income
loss was found to have a skew of 0.77 and kurtosis of 1.11. Expected
Eviction or Home Foreclosure was found to have a skew of -0.01 and a
kurtosis of -0.56. The distribution of those who ``Delayed Medical
Care'' was calculated to have a skew of 0.2 and a kurtosis of -0.17.

It is interesting to note that averages for all three of the variables
fell between 30 and 40 percent. Further illustrating the severity of
Covid-19's impact, the minimum value of respondents that expected income
loss across states was 20.6\%. Even higher, the minimum value of
respondents that had delayed seeking medical care across states was
30.3\%.

\includegraphics[width=0.33\linewidth]{Vaccine_Hesitancy_Final.FigueroaHollly_files/figure-latex/curve plots for normality on state stability responses-1}
\includegraphics[width=0.33\linewidth]{Vaccine_Hesitancy_Final.FigueroaHollly_files/figure-latex/curve plots for normality on state stability responses-2}
\includegraphics[width=0.33\linewidth]{Vaccine_Hesitancy_Final.FigueroaHollly_files/figure-latex/curve plots for normality on state stability responses-3}

\textbf{County Variable Correlational Analyses}

The first relationships explored by correlation are between the county
level data. Two matrices are used to illustrate the correlations between
county race and county hesitancy estimations. The first specifically
looks at ``Hesitant'' estimates while the second looks at ``strongly
hesitant'' estimations. There almost no difference between. These
correlations were conducted in the Spearman method due to non-normality
of race distributions. The highest correlations to hesitancy estimations
are with the Hispanic and Asian populations. Hispanic and Asian
populations were negatively correlated, meaning an increase in either of
these populations within a given county, decreases in hesitancy are also
found. Hispanic populations were correlated with ``hesitant'' estimates
at -0.27 rho, and ``strongly hesitant'' at - 0.25 rho, respectively.
Both were significant with p-values \textless{} 2.2e-16 suggesting these
correlations are unlikely to change with further sampling. Asian
populations were correlated with ``hesitant'' estimates at -0.33 rho,
and ``strongly hesitant'' at -0.32 rho. Both again, with p-values
\textless{} 2.2e-16, suggesting larger sampling would not change the
correlation finding.

\includegraphics[width=0.5\linewidth]{Vaccine_Hesitancy_Final.FigueroaHollly_files/figure-latex/COUNTY LEVEL ANALYSES-1}
\includegraphics[width=0.5\linewidth]{Vaccine_Hesitancy_Final.FigueroaHollly_files/figure-latex/COUNTY LEVEL ANALYSES-2}

The table below illustrates Spearman's correlations between republican
votes and hesitancy data. Further analysis show that republican votes
are borderline moderately correlated. Presidential votes were correlated
at a rho of 0.29 with Hesitant estimates and a rho of 0.30 with Strongly
Hesitant estimations by county. Both measures for rho were found to be
significant with p-values far below 0.0001. This suggests that the more
county votes cast for republican candidate, the more a county was
estimated to be hesitant to vaccination.

\includegraphics[width=0.55\linewidth]{Vaccine_Hesitancy_Final.FigueroaHollly_files/figure-latex/unnamed-chunk-2-1}

\textbf{State Variable Correlational Analyses}

The table below offers ample information on our variables of interest at
the state level. Focusing on correlations to ``willingness'' to
vaccinate, we find significant results with most of the variables. The
highest correlation is the percentage of votes, ``trump votes'', which
is negatively correlated with willingness to vaccinate at -0.80 with a
p-value below 0.0001. The suggests that the higher proportion of votes a
state had that were awarded to the republican candidate, the lower the
percentage was of people willing to get vaccinations. The next highest
finding was with the percentage of people who expected eviction. With a
negative correlation of -0.47 and p-value under 0.0001, the result is
significant. This suggests states with more housing instability have
lower percentages of people willing to vaccinate. Lastly, delayed
medical care has a moderate correlation to willingness to vaccinate at
0.31 and a p-value at or under 0.05. This suggests states with more
people that delayed their medical care needs, had higher percentages of
people willing to vaccinate.

\newline

\includegraphics[width=0.75\linewidth]{Vaccine_Hesitancy_Final.FigueroaHollly_files/figure-latex/STATE LEVEL ANALYSES-1}

Because these finding were conducted with ggpairs, from the GGally
package, the default method is a Pearson's Correlation. As the
distributions in the table above show signs of skew and non-normal
distribution. Follow-up correlations using non-parametric methods were
conducted to make for a more robust and conclusive result. The two
negative correlations returned were slightly higher, at -0.48 for
``Expected Eviction'' and -0.82 for ``Republican Votes''. The p-values
for both of these results stayed below 0.001. The rho correlation found
between ``Delayed Medical Care'' lowered to 0.24 and the p-value rose to
0.087. Making this final correlation no-longer statistically
significant.

\hypertarget{implications}{%
\subsection{Implications}\label{implications}}

A strong and significant relationship was found between election data
and vaccine views at the state level and county level. To have a factor
such as political party -80\% correlated to being willing to vaccinate,
is no small concern. At worst, the implication is that government
leaders have used the topic of pandemic solutions to fuel distrust and
inner-party cohesion. The cost of which, has yet to be fully seen. At
best, conflicting messages from leadership and qualified institutions,
such as the CDC, have opened the public up to distrust along party
lines, without intention. The benefit of finding a relationships to
election data is that the art of ``reaching'' voters through messaging
is vast. The knowledge that a relationship exists could be used to help
tailor marketing by political identity to increase vaccinations.

The analysis showed that an average 30-40\% of people across the US
reported hardships during the pandemic. Despite this, the average
percent of people across states willing to vaccinate was under 50\%
leaving many at risk. It would be reasonable to assume that people with
less stability of housing, income, and interrupted medical care, would
be more likely to seek vaccination as a form of increased safety and
stability. However, this was not the finding of the analyses for all
instability measures. Income instability appears to have little-to-no
relation to vaccination views. Meanwhile, populations more under threat
of eviction were associated with being less willing to vaccinate. This
may indicate a confounding variable at work not included in this
analysis. The positive correlation for vaccinating with delayed medical
care was the only positive correlation of note among the ``hardship''
variables. The significance of this finding was lost with when
calculated more robustly.

\hypertarget{limitations}{%
\subsection{Limitations}\label{limitations}}

As correlation was the focus of these analysis, it is important to note
that no correlation can indicate causation. The measures presented here
can, however, serve further analysis. A major limitation of this work is
that election data was limited to the presidential office only.
Exploring other offices and the proportion of offices held by party
would go much further. Exploring known correlates of party affiliation
would also lend clarity to these findings. Similarly, correlates of
income and housing stability, such as education, could lend clarity as
well. A major limitation of the data regarding income loss and housing
instability is that the survey data does not explicitly capture these
hardships as being associated with the pandemic. It cannot be assumed if
respondents view or associate these changes to the pandemic. This could
be a distinguishing factor in how much value is placed on vaccination,
if seen as step towards stability.

\hypertarget{concluding-remarks}{%
\subsection{Concluding Remarks}\label{concluding-remarks}}

As the current administration has announced a goal of 70\% vaccination,
much work is to be done. As vaccinations increase, there is hope that
more hesitant-minded people will be reassured enough to sign up. In the
meantime, results here suggest that leadership communication to the
public is not to be taken lightly. For the moment, information such as
the findings here can offer local governments a template for who needs
encouragement and who best to offer it.

\hypertarget{references}{%
\subsection{References}\label{references}}

Estimation of total mortality due to COVID-19. (2021, May 26). Institute
for Health Metrics and Evaluation.
\url{http://www.healthdata.org/special-analysis/estimation-excess-mortality-due-covid-19-and-scalars-reported-covid-19-deaths}

Grossman, G. (2020, September 29). Political partisanship influences
behavioral responses to governors' recommendations for COVID-19
prevention in the United States. PNAS.
\url{https://www.pnas.org/content/117/39/24144}

Schwartz, J. L. (2012, January 1). New Media, Old Messages: Themes in
the History of Vaccine Hesitancy and Refusal. Journal of Ethics
\textbar{} American Medical Association.
\url{https://journalofethics.ama-assn.org/article/new-media-old-messages-themes-history-vaccine-hesitancy-and-refusal/2012-01}

\end{document}
